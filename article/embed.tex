\documentclass{article}

\title{Hilbert Curves for Unequal Dimensions}
\author{Andrew Dolgert}
\date{\today}
\begin{document}
\maketitle
\begin{abstract}
Given a lattice-continuous Hilbert curve defined for $n$-dimensions on a hypercube with
extent $2^k$, it is always possible to
create a lattice-continuous Hilbert curve in $n$-dimensions with unequal extents that
are powers of two.
This article shows how to construct such a Hilbert curve.
\end{abstract}



\subsection{Affine Transforms on Directed Edges of the Hypercube}
 \begin{theorem}
  Let $n \ge 1$. The set of affine transforms $\{S_{e,\delta} : e \in \mathbb{F}_2^n,\, \delta \in \mathbb{Z}_n\}$
  is in bijection with the set of directed edges of the $n$-dimensional hypercube.
  \end{theorem}

  \begin{proof}
  Let $V = \mathbb{F}_2^n$ be the vector space over $\mathbb{F}_2$ with standard basis
  $\{\mathbf{e}_0, \ldots, \mathbf{e}_{n-1}\}$.
  The vertices of the $n$-cube are the elements of $V$, and two vertices are adjacent
  if and only if they differ by some $\mathbf{e}_a$.

  A \emph{directed edge} is an ordered pair $(u, u + \mathbf{e}_a)$ where $u \in V$ and $a \in \mathbb{Z}_n$.

  \medskip
  \noindent\textbf{Counting.}
  There are $2^n$ choices for $u$ and $n$ choices for $a$, giving $n \cdot 2^n$ directed edges.
  There are $2^n$ choices for $e$ and $n$ choices for $\delta$, giving $n \cdot 2^n$ transforms.

  \medskip
  \noindent\textbf{The transform.}
  Let $\rho \in \mathrm{Sym}(\mathbb{Z}_n)$ be the cyclic permutation $\rho(a) = (a+1) \bmod n$,
  acting on $V$ by $(\rho \cdot v)(a) = v(\rho^{-1}(a))$.
  Then $\rho \cdot \mathbf{e}_j = \mathbf{e}_{(j+1) \bmod n}$, and hence
  $\rho^{-\delta} \cdot \mathbf{e}_j = \mathbf{e}_{(j-\delta) \bmod n}$.

  Define $S_{e,\delta} : V \to V$ by
  \[
  S_{e,\delta}(x) = \rho^{-\delta} \cdot x + e.
  \]

  \medskip
  \noindent\textbf{Reference edge.}
  Fix the directed edge $R = (0, \mathbf{e}_{n-1})$ as the reference.
  Under $S_{e,\delta}$:
  \begin{align*}
  S_{e,\delta}(0) &= e, \\
  S_{e,\delta}(\mathbf{e}_{n-1}) &= \mathbf{e}_{(n-1-\delta) \bmod n} + e.
  \end{align*}
  Thus $S_{e,\delta}$ maps $R$ to the directed edge from $e$ along axis $(n-1-\delta) \bmod n$.

  \medskip
  \noindent\textbf{Surjectivity.}
  Given any directed edge $(u, u + \mathbf{e}_a)$, choose
  \[
  e = u, \qquad \delta = (n - 1 - a) \bmod n.
  \]
  Then $S_{e,\delta}(R) = (u, u + \mathbf{e}_a)$.

  For the reverse orientation $(u + \mathbf{e}_a, u)$, choose
  \[
  e = u + \mathbf{e}_a, \qquad \delta = (n - 1 - a) \bmod n.
  \]
  Then:
  \begin{align*}
  S_{e,\delta}(0) &= u + \mathbf{e}_a, \\
  S_{e,\delta}(\mathbf{e}_{n-1}) &= (u + \mathbf{e}_a) + \mathbf{e}_a = u.
  \end{align*}

  \medskip
  \noindent\textbf{Injectivity.}
  The map $S_{e,\delta} \mapsto S_{e,\delta}(R)$ sends distinct pairs $(e, \delta)$ to distinct
  directed edges, since $e$ determines the starting vertex and $\delta$ determines the axis.

  \medskip
  \noindent\textbf{Conclusion.}
  The map $S_{e,\delta} \mapsto S_{e,\delta}(R)$ is a bijection between
  $\{S_{e,\delta}\}$ and directed edges of the $n$-cube.
  \end{proof}

  \begin{lemma}[Interpretation for Hilbert curves]
  The state $(e, \delta)$ of a Hilbert curve encodes exactly the information needed to specify
  an entry vertex and an exit direction. The bijection above shows this encoding is tight:
  no redundancy, no missing configurations.
  \end{lemma}

\end{document}
