\documentclass[11pt]{article}
\usepackage{amsmath,amssymb,amsthm}
\usepackage{tikz}
\usetikzlibrary{positioning,calc}
\usepackage{booktabs}
\usepackage{enumitem}

\newtheorem{theorem}{Theorem}[section]
\newtheorem{lemma}[theorem]{Lemma}
\newtheorem{proposition}[theorem]{Proposition}
\newtheorem{corollary}[theorem]{Corollary}
\theoremstyle{definition}
\newtheorem{definition}[theorem]{Definition}
\newtheorem{example}[theorem]{Example}
\theoremstyle{remark}
\newtheorem{remark}[theorem]{Remark}

\newcommand{\NN}{\mathbb{N}}
\newcommand{\ZZ}{\mathbb{Z}}
\newcommand{\Sym}{\mathrm{Sym}}

\newcommand{\Gray}{\mathsf{G}}
\newcommand{\entry}{\mathsf{entry}}
\newcommand{\exitp}{\mathsf{exit}}
\newcommand{\embed}{\uparrow}

\newcommand{\normone}[1]{\left\lVert #1 \right\rVert_1}

\title{Hilbert Curves via Oriented Gray Codes}
\author{}
\date{}

\begin{document}
\maketitle

\begin{abstract}
We present a framework for Hilbert lattice curves based on Gray codes and orientations.
The key viewpoint is that a Hilbert curve is a hierarchy of \emph{oriented binary reflected Gray codes} (BRGCs),
glued along shared faces.
This framework extends naturally to anisotropic dyadic boxes (unequal side lengths)
once we specify (i) an ordering of axes and (ii) an embedding rule that extends orientations when new axes activate.
\end{abstract}

%=============================================================================
\section{The hypercube and the binary reflected Gray code}
%=============================================================================

\begin{definition}[$k$-cube corners]
For $k\ge 1$, the \emph{$k$-cube corner set} is
\[
V_k := \{0,1\}^k,
\]
whose elements we call \emph{corners} or \emph{vertices}.
\end{definition}

\begin{definition}[Adjacency and Hamming distance]
Two corners $u,v\in V_k$ are \emph{adjacent} if they differ in exactly one coordinate.
Equivalently, their \emph{Hamming distance} is $d_H(u,v)=1$.
\end{definition}

\begin{definition}[Gray code]
A \emph{Gray code} on $V_k$ is a sequence
\[
(c_0,c_1,\dots,c_{2^k-1})\quad\text{with }c_i\in V_k
\]
that visits each corner exactly once and satisfies $d_H(c_i,c_{i+1})=1$ for all $i$.
\end{definition}

We will use a particular Gray code: the \emph{binary reflected Gray code} (BRGC).
We define it inductively (no bit arithmetic is needed).

\begin{definition}[Binary reflected Gray code (BRGC)]\label{def:brgc}
Define $\Gray_1:\{0,1\}\to V_1$ by $\Gray_1(0)=0$ and $\Gray_1(1)=1$.
Assume $\Gray_k:\{0,\dots,2^k-1\}\to V_k$ is defined.
Define $\Gray_{k+1}:\{0,\dots,2^{k+1}-1\}\to V_{k+1}$ by
\[
\Gray_{k+1}(i) :=
\begin{cases}
0\,\Gray_k(i) & 0\le i < 2^k,\\[2pt]
1\,\Gray_k(2^{k+1}-1-i) & 2^k \le i < 2^{k+1},
\end{cases}
\]
where $b\,x$ denotes concatenation of a leading bit $b\in\{0,1\}$ with a $k$-bit string $x\in V_k$.
\end{definition}

\begin{lemma}[BRGC adjacency]\label{lem:brgc-adj}
For every $k\ge 1$, the sequence $\bigl(\Gray_k(0),\Gray_k(1),\dots,\Gray_k(2^k-1)\bigr)$ is a Gray code on $V_k$.
\end{lemma}

\begin{proof}
By induction on $k$.
For $k=1$ this is immediate.
Assume the claim holds for $k$.
The first half of $\Gray_{k+1}$ is $0\,\Gray_k(0),\dots,0\,\Gray_k(2^k-1)$, which is adjacent step-by-step by induction.
The second half is $1\,\Gray_k(2^k-1),\dots,1\,\Gray_k(0)$, again adjacent step-by-step by induction.
At the seam, the last element of the first half is $0\,\Gray_k(2^k-1)$ and the first element of the second half is
$1\,\Gray_k(2^k-1)$, which differ in exactly the leading coordinate.
\end{proof}

\begin{definition}[The flip coordinate]\label{def:g}
For $k\ge 1$ and $0\le i < 2^k-1$, define $g_k(i)\in\{0,\dots,k-1\}$ to be the \emph{unique} coordinate index
for which $\Gray_k(i)$ and $\Gray_k(i+1)$ differ.
(Existence and uniqueness follow from Lemma~\ref{lem:brgc-adj}.)
\end{definition}

\begin{example}[BRGC for $k=2$]\label{ex:brgc2}
\[
\Gray_2(0)=00,\quad
\Gray_2(1)=01,\quad
\Gray_2(2)=11,\quad
\Gray_2(3)=10.
\]
\end{example}

\begin{example}[BRGC for $k=3$]\label{ex:brgc3}
\[
\begin{array}{c|c}
i & \Gray_3(i) \\\midrule
0 & 000 \\
1 & 001 \\
2 & 011 \\
3 & 010 \\
4 & 110 \\
5 & 111 \\
6 & 101 \\
7 & 100 \\
\end{array}
\]
\end{example}

%=============================================================================
\section{Orientations and oriented Gray codes}
%=============================================================================

The Hilbert recursion places copies of a lower-order curve inside each subcube.
Those copies may be rotated and reflected; we model this combinatorially on corners.

\begin{definition}[Axis permutation action]
Let $\pi\in\Sym(k)$ be a permutation of $\{0,\dots,k-1\}$.
Its action on $v\in V_k$ is defined by permuting coordinates:
\[
(\pi\cdot v)_a := v_{\pi^{-1}(a)}.
\]
\end{definition}

\begin{definition}[Orientation map]\label{def:orientation}
An \emph{orientation} on $V_k$ is specified by a pair $(e,\pi)$ with $e\in V_k$ and $\pi\in\Sym(k)$.
It acts on corners by
\[
\mathcal{O}_{(e,\pi)}(v) := \pi\cdot v \oplus e,
\]
where $\oplus$ denotes coordinatewise XOR on $\{0,1\}$.
\end{definition}

\begin{definition}[Oriented BRGC]\label{def:oriented-brgc}
Given an orientation $(e,\pi)$ on $V_k$, the associated \emph{oriented BRGC} is the sequence
\[
c_i := \mathcal{O}_{(e,\pi)}\bigl(\Gray_k(i)\bigr),\qquad i=0,\dots,2^k-1.
\]
\end{definition}

\begin{lemma}[Orientations preserve adjacency]\label{lem:orient-adj}
If $u,v\in V_k$ are adjacent, then $\mathcal{O}_{(e,\pi)}(u)$ and $\mathcal{O}_{(e,\pi)}(v)$ are adjacent.
In particular, every oriented BRGC (Definition~\ref{def:oriented-brgc}) is a Gray code on $V_k$.
\end{lemma}

\begin{proof}
Adjacency means $u$ and $v$ differ in exactly one coordinate.
Permuting coordinates preserves that property, and XOR with a fixed $e$ flips the same coordinates in both points,
so it also preserves the set of differing coordinates.
\end{proof}

%=============================================================================
\section{Generators and the gluing condition}
%=============================================================================

At a given recursion step, we subdivide a $k$-dimensional cell into $2^k$ children and visit them in BRGC order.
A \emph{generator} specifies how each child is oriented.

\begin{definition}[Generator for dimension $k$]\label{def:generator}
A \emph{$k$-dimensional generator} is a map
\[
\mu_k:\{0,\dots,2^k-1\}\to V_k\times \Sym(k),
\qquad \mu_k(i)=(e(i),\pi(i)),
\]
assigning an orientation to the $i$-th child in the traversal order.
\end{definition}

\begin{definition}[Entry and exit corners]\label{def:entry-exit}
Fix $k$ and a generator $\mu_k$.
For child $i$, define
\[
\entry(i):=e(i),
\qquad
\exitp(i):=\mathcal{O}_{(e(i),\pi(i))}\bigl(\Gray_k(2^k-1)\bigr).
\]
That is, $\entry(i)$ is the first corner of the oriented BRGC inside the child,
and $\exitp(i)$ is the last corner of that oriented BRGC.
\end{definition}

\begin{definition}[Gluing condition]\label{def:gluing}
A generator $\mu_k$ satisfies the \emph{gluing condition} if for all $0\le i<2^k-1$,
\[
\exitp(i)\ \oplus\ \mathbf{e}_{g_k(i)} \;=\; \entry(i+1),
\]
where $g_k(i)$ is the flip coordinate from Definition~\ref{def:g} and $\mathbf{e}_a\in V_k$ is the $a$-th unit vector.
\end{definition}

\begin{remark}
The gluing condition says: if consecutive children differ in the BRGC along local coordinate $g_k(i)$,
then the end of child $i$ and the start of child $i+1$ are also adjacent along that same local coordinate.
This is the only local constraint needed to ensure lattice continuity at child boundaries.
\end{remark}

%=============================================================================
\section{Anisotropic dyadic boxes, levels, and active axes}
%=============================================================================

For equal side lengths $2^m$ in each dimension, all axes participate at every recursion step.
For anisotropic boxes, different axes become active at different levels.

\begin{definition}[Dyadic box]\label{def:box}
Fix $n\ge 1$ and extents $m=(m_0,\dots,m_{n-1})\in\NN^n$.
Define the \emph{dyadic box}
\[
P(m) := \prod_{j=0}^{n-1}\{0,1,\dots,2^{m_j}-1\}\subset \ZZ^n,
\]
and write $M:=\sum_{j=0}^{n-1} m_j$ and $m_{\max}:=\max_j m_j$.
\end{definition}

\begin{definition}[Axis priority order]\label{def:axis-order}
Fix the extents $m$.
Define a total order $\prec$ on axes $\{0,\dots,n-1\}$ by
\[
j\prec j' \quad\Longleftrightarrow\quad (m_j,j)\text{ is lexicographically smaller than }(m_{j'},j').
\]
(Shorter extents first; ties broken by axis index.)
\end{definition}

\begin{definition}[Levels and active axes]\label{def:levels-active}
We index levels by integers $s\in\{m_{\max},m_{\max}-1,\dots,1\}$, where $s=m_{\max}$ is the coarsest split
and $s=1$ is the finest split.

At level $s$, the \emph{active axis set} is
\[
A_s := \{j\in\{0,\dots,n-1\} : m_j\ge s\}.
\]
Let $k_s:=|A_s|$ and let
\[
A_s=[a^{(s)}_0,\dots,a^{(s)}_{k_s-1}]
\]
denote the active axes listed in increasing $\prec$-order (Definition~\ref{def:axis-order}).
\end{definition}

\begin{proposition}[Monotone activation]\label{prop:monotone}
For $2\le s\le m_{\max}$ we have $A_s\subseteq A_{s-1}$ as sets, hence $k_s\le k_{s-1}$.
Equivalently: as we move from coarse to fine (decreasing $s$), axes may activate but never deactivate.
\end{proposition}

\begin{definition}[Inclusion map between active-axis lists]\label{def:inclusion}
Whenever $2\le s\le m_{\max}$, the ordered list $A_s$ is an order-preserving subsequence of $A_{s-1}$.
Define the (unique) order-preserving injection
\[
\iota_s:\{0,\dots,k_s-1\}\hookrightarrow \{0,\dots,k_{s-1}-1\}
\]
by the rule
\[
a^{(s)}_t = a^{(s-1)}_{\iota_s(t)}\qquad\text{for all }t.
\]
\end{definition}

\begin{example}[A $4\times 2$ box]\label{ex:4x2}
Let $n=2$ and $m=(2,1)$, so $P(m)=\{0,\dots,3\}\times\{0,1\}$.
Here $m_{\max}=2$.
The axis priority order has $1\prec 0$ (since $m_1=1<m_0=2$).
Thus:
\[
A_2=[0]\quad(k_2=1),\qquad
A_1=[1,0]\quad(k_1=2).
\]
At the coarsest level $s=2$ only axis $0$ is active; at the finer level $s=1$ axis $1$ activates and is inserted \emph{before} axis $0$ in the ordered list.
\end{example}

%=============================================================================
\section{Embedding orientations when new axes activate}
%=============================================================================

When moving from level $s$ to the finer level $s-1$, the active dimension may increase from $k_s$ to $k_{s-1}$.
We embed corner labels and orientations using the inclusion map $\iota_s$ (Definition~\ref{def:inclusion}).

\begin{definition}[Embedding of corners]\label{def:embed-corners}
For $2\le s\le m_{\max}$ define the embedding
\[
E_s:V_{k_s}\to V_{k_{s-1}}
\]
by
\[
\bigl(E_s(u)\bigr)_{\iota_s(t)} := u_t \quad\text{for }t=0,\dots,k_s-1,
\qquad
\bigl(E_s(u)\bigr)_q := 0 \quad\text{for }q\notin \mathrm{im}(\iota_s).
\]
In words: keep the bits on shared axes, and set newly activated coordinates to $0$.
\end{definition}

\begin{definition}[Embedding of orientations]\label{def:embed-orientation}
Let $(e,\pi)$ be an orientation on $V_{k_s}$.
Define its embedded orientation $(e^\embed,\pi^\embed)$ on $V_{k_{s-1}}$ by:
\[
e^\embed := E_s(e),
\]
and
\[
\pi^\embed(q) :=
\begin{cases}
\iota_s\bigl(\pi(t)\bigr) & \text{if }q=\iota_s(t)\text{ for some }t\in\{0,\dots,k_s-1\},\\[2pt]
q & \text{if }q\notin \mathrm{im}(\iota_s).
\end{cases}
\]
That is: $\pi$ is transported to the larger coordinate system via $\iota_s$, and new coordinates are fixed.
\end{definition}

\begin{lemma}[Embedding preserves adjacency]\label{lem:embed-adj}
If $u,v\in V_{k_s}$ are adjacent (differ in local coordinate $t$),
then $E_s(u)$ and $E_s(v)$ are adjacent in $V_{k_{s-1}}$ (differ in local coordinate $\iota_s(t)$).
\end{lemma}

\begin{proof}
All newly activated coordinates are $0$ for both $E_s(u)$ and $E_s(v)$.
On shared coordinates, $u$ and $v$ differ in exactly one position $t$, which is carried to position $\iota_s(t)$.
\end{proof}

\begin{proposition}[Embedding preserves the gluing condition]\label{prop:embed-gluing}
Let $\mu_{k_s}$ be a generator for dimension $k_s$ satisfying the gluing condition (Definition~\ref{def:gluing}).
Transport each child orientation by Definition~\ref{def:embed-orientation} to obtain an induced generator in dimension $k_{s-1}$.
Then the gluing condition continues to hold in the larger coordinate system (with $g_{k_s}(i)$ replaced by $\iota_s(g_{k_s}(i))$).
\end{proposition}

\begin{proof}
By Lemma~\ref{lem:embed-adj}, adjacency of endpoints in $V_{k_s}$ becomes adjacency in $V_{k_{s-1}}$ at the embedded coordinate.
\end{proof}

\begin{remark}[What “extend $\pi$ by identity” really means]
If one thinks of $\pi$ as acting on \emph{physical axes}, then activating a new axis truly corresponds to “extend by the identity.”
In the present mixed-dimensional setup, however, $\pi$ acts on the \emph{local coordinate indices} $\{0,\dots,k_s-1\}$,
so we must also specify how old local indices are identified with new ones. That identification is exactly the inclusion map $\iota_s$.
\end{remark}

%=============================================================================
\section{A seam lemma for dyadic scaling}
%=============================================================================

The gluing condition is stated on \emph{corner labels}. To translate it into a unit lattice step,
we use the following elementary observation about dyadic splits.

\begin{lemma}[Dyadic seam is a unit step]\label{lem:dyadic-seam}
Fix $s\ge 1$ and consider a split of an interval of length $2^s$ into two halves of length $2^{s-1}$.
Let $q,q'\in\{0,1\}$ denote the lower/upper half and let $c,c'\in\{0,1\}$ denote the lower/upper \emph{endpoint within that half}.
Define the corresponding lattice coordinates by
\[
\Phi_s(q,c):= q\cdot 2^{s-1} + c\cdot(2^{s-1}-1).
\]
If $(q',c')=(q\oplus 1,\,c\oplus 1)$, then $|\Phi_s(q',c')-\Phi_s(q,c)|=1$.
\end{lemma}

\begin{proof}
A direct check gives the two possibilities $(q,c)=(0,1)\mapsto(1,0)$ and $(1,0)\mapsto(0,1)$, yielding $\pm 1$.
\end{proof}

%=============================================================================
\section{Continuity for the active-axis construction (statement level)}
%=============================================================================

We state the continuity claim in the form used by the activation proof:
continuity reduces to checking seams at the first level where two indices differ.

\begin{theorem}[Lattice continuity (active axes)]\label{thm:continuity}
Let $m\in\NN^n$ and consider a hierarchical traversal of $P(m)$ built level-by-level as follows:
at each level $s$ the current cell is split along the active axes $A_s$ into $2^{k_s}$ children,
the children are visited in BRGC order, and each child receives an orientation so that the local gluing condition holds.
When moving from level $s$ to $s-1$ and new axes activate, orientations are embedded via Definitions
\ref{def:embed-corners}--\ref{def:embed-orientation}.

Then consecutive points in the traversal are Manhattan neighbors:
\[
\normone{H_m(h+1)-H_m(h)}=1\qquad\text{for all }0\le h<2^M-1.
\]
\end{theorem}

\begin{proof}[Proof sketch]
Let $h$ and $h+1$ be consecutive indices and let $s^{*}$ be the finest level at which their mixed-radix digits first differ.
Equivalently, below level $s^{*}$ all finer digits overflow from their maximum to $0$, and the digit at level $s^{*}$ advances to the next child in BRGC order.

Thus $H_m(h)$ is the \emph{exit point} of one level-$s^{*}$ child and $H_m(h+1)$ is the \emph{entry point} of the next child.
By the gluing condition, their corner labels differ in exactly one local coordinate, namely the flip coordinate $g_{k_{s^{*}}}(i)$.
At the physical grid scale $2^{s^{*}-1}$, adjacent children differ by an offset of $2^{s^{*}-1}$ along that active axis,
and the gluing condition flips the corresponding endpoint bit. Lemma~\ref{lem:dyadic-seam} then gives a unit step in $\ZZ^n$.
Embedding across activation boundaries preserves which \emph{physical} axis this seam lies on (Definition~\ref{def:inclusion}).
All other coordinates coincide, so the $\ell^1$ distance is $1$.
\end{proof}

%=============================================================================
\section{Summary}
%=============================================================================

The framework reduces the mechanics of Hilbert lattice continuity to three ingredients:
\begin{enumerate}[nosep]
\item the BRGC as the canonical order of children at each split;
\item orientations as adjacency-preserving re-labellings of corners;
\item the gluing condition, plus an embedding rule that keeps axis identity consistent when the active dimension changes.
\end{enumerate}

%=============================================================================
\appendix
\section{Optional: closed forms (reference only)}
%=============================================================================

The present development deliberately avoids closed bit formulas.
When one later wishes to implement the BRGC by arithmetic on integers, it admits standard closed forms,
and Hamilton's $T$-transform is one way to represent orientations efficiently.
Those details are not needed for the structural arguments above.

\end{document}
