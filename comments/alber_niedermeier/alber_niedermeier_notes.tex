
% Personal notes / rewrite of:
%   J. Alber and R. Niedermeier, "On Multidimensional Curves with Hilbert Property",
%   Theory of Computing Systems 33, 295–312 (2000).
%
% Goal: keep the mathematics, but rewrite notation and prose in a more "modern / direct"
% style (closer to article_clean.tex), with a small dictionary to the original paper.

\documentclass[11pt]{article}

\usepackage{amsmath,amssymb,amsthm,mathtools}
\usepackage{enumitem}
\usepackage{hyperref}

\hypersetup{colorlinks=true,linkcolor=blue,citecolor=blue,urlcolor=blue}

% -------------------- Theorem environments --------------------
\theoremstyle{plain}
\newtheorem{theorem}{Theorem}
\newtheorem{lemma}{Lemma}

\theoremstyle{definition}
\newtheorem{definition}{Definition}
\newtheorem{remark}{Remark}

% -------------------- Notation --------------------
\newcommand{\F}{\mathbb{F}_2}
\newcommand{\V}[1]{\F^{#1}}
\newcommand{\bits}{\{0,1\}}
\newcommand{\Z}{\mathbb{Z}}
\newcommand{\id}{\mathrm{id}}
\newcommand{\Sym}{\mathrm{Sym}}
\newcommand{\hd}{d_h} % Hamming distance
\newcommand{\ord}{\operatorname{ord}}

% L_p metrics (we only really use p in {1,2,infty} here)
\newcommand{\dOne}{d_1}
\newcommand{\dTwo}{d_2}
\newcommand{\dInf}{d_\infty}

% Convenience: r-dimensional dyadic cube grid of order k
\newcommand{\Grid}[2]{\{0,\dots,2^{#1}-1\}^{#2}}

\title{Notes on Alber--Niedermeier (2000): Multidimensional Curves with Hilbert Property}
\author{personal rewrite in the style of \texttt{article\_clean.tex}}
\date{}

\begin{document}
\maketitle

\section{What the paper is doing (in one paragraph)}

Fix a dimension $r\ge 2$.
We care about indexing the integer grid points of an $r$-dimensional cube
$\Grid{k}{r}$ (side length $2^k$) by a single integer in such a way that consecutive
indices land on adjacent grid points (a Hamiltonian path of the grid graph).
A classical way to do this is the 2D Hilbert curve, which is recursively built from
rotated/reflected copies of itself.
Alber--Niedermeier formalize this kind of ``Hilbert-like self-similarity'' in arbitrary
dimension using only:
(i) an order-$1$ \emph{generator} (a path through the $2^r$ corners), and
(ii) a fixed list of cube symmetries that tell you how each recursive subcube is oriented.
They call such families \emph{CSSIs} and the continuous ones \emph{CHPs}.
The key structural message is: in this rigid self-similar setting, questions about the
whole infinite family reduce to checking the small generating data.

\section{Quick dictionary to the original paper}

The original paper uses $1$-based indexing and labels corners by $\{1,\dots,2^r\}$.
I will mostly use $0$-based indices and identify corners with bit-vectors in $\V{r}$.

\begin{itemize}[leftmargin=*,itemsep=2pt]
\item Paper: grid of order $k$ has side length $2^k$ and $n^r=(2^k)^r$ points.\\
Notes: grid is $\Grid{k}{r}=\{0,\dots,2^k-1\}^r$ and has $2^{kr}$ points.
\item Paper: an indexing is $C:\{1,\dots,(2^k)^r\}\to \{1,\dots,2^k\}^r$.\\
Notes: an indexing is $C:[0,2^{kr})\to \Grid{k}{r}$.
\item Paper: corners are $V_r=\{x_1\cdots x_r\mid x_i\in\bits\}$.\\
Notes: corners are $\V{r}$, written as bit-vectors.
\item Paper: a corner labeling is $\mathcal I:V_r\to\{1,\dots,2^r\}$.\\
Notes: a corner labeling is $I:\V{r}\to[0,2^r)$.
\item Paper: $W_{\mathcal I}\subseteq \Sym(2^r)$ is the ``symmetry group'' w.r.t.~$\mathcal I$.\\
Notes: I view $W_I$ as the automorphism group of the hypercube on $\V{r}$,
i.e., all maps $x\mapsto Px\oplus a$ (permutation of coordinates + bit-flips).
When you insist on labels $[0,2^r)$, you recover the paper's definition by conjugation.
\end{itemize}

\section{Preliminaries: grids, indexings, curves}

\subsection{Indexings and the curve condition}

Fix $r\ge 1$ and $k\ge 1$.
Let $\Grid{k}{r}$ be the set of lattice points in the $r$-cube of side length $2^k$.
We view $\Grid{k}{r}$ as a grid graph: $p,q\in\Grid{k}{r}$ are adjacent if
$\|p-q\|_1=1$.

\begin{definition}[Indexing and curve]
An \emph{$r$-dimensional indexing of order $k$} is a bijection
\[
C:[0,2^{kr})\to \Grid{k}{r}.
\]
It is a \emph{curve} (lattice-continuous) if
\[
\|C(t+1)-C(t)\|_1=1\qquad\text{for all }t\in[0,2^{kr}-1).
\]
\end{definition}

\subsection{Corners and Gray codes}

The $2^r$ corners of an $r$-cube can be identified with $\V{r}$.
Two corners are adjacent (share an edge) iff they differ in exactly one bit.
So an order-$1$ \emph{continuous} indexing is exactly a Hamiltonian path on the
$r$-dimensional hypercube graph $Q_r$ (a Gray code).

This viewpoint is useful because Alber--Niedermeier really build everything from an
order-$1$ object.

\section{CSSIs: self-similar indexings from a generator + symmetries}

This is the core formalism of the paper.

\subsection{Corner labelings and the symmetry group}

A \emph{corner labeling} is just a way to name the $2^r$ corners.

Formally, fix a bijection $I:\V{r}\to[0,2^r)$.
This induces an adjacency predicate on labels:
\[
n_I(a,b)=
\begin{cases}
1 & \text{if } \hd\!\bigl(I^{-1}(a),I^{-1}(b)\bigr)=1,\\
0 & \text{otherwise,}
\end{cases}
\]
where $\hd$ is Hamming distance.

Then define $W_I$ to be the set of all permutations of $[0,2^r)$ that preserve this
adjacency relation:
\[
W_I\coloneqq\{\pi\in\Sym(2^r)\mid n_I(a,b)=n_I(\pi(a),\pi(b))\ \forall a,b\}.
\]
Intuition: elements of $W_I$ are exactly the rotations/reflections of the $r$-cube,
expressed as permutations of corner labels.

\paragraph{From corner permutations to grid symmetries.}
Each $\pi\in W_I$ acts on corners by conjugation,
\[
\pi_I \;\coloneqq\; I^{-1}\circ \pi \circ I:\V{r}\to\V{r},
\]
and this corner action extends in the obvious way to an isometry of the full grid
$\Grid{k}{r}$ (permute axes and/or reflect along axes).
The paper writes this induced grid map as $(\pi:I)$ and then freely identifies $\pi$
with $(\pi:I)$.
I'll also do that when it is clear from context.

\subsection{Partitioning the order-$k$ cube into $2^r$ subcubes}

For $k\ge 2$, partition $\Grid{k}{r}$ into $2^r$ disjoint subcubes of order $k-1$,
indexed by corners $x\in \V{r}$.

Define the lower-left (minimal) corner of subcube $x$ by
\[
p^{(k)}(x)\coloneqq 2^{k-1}\,x \in \Grid{k}{r}.
\]
So subcube $x$ is exactly
\[
p^{(k)}(x) + \Grid{k-1}{r}.
\]

\subsection{The basic construction step}

Suppose we already have an indexing $C_{k-1}$ of order $k-1$.
Fix a corner labeling $I$ and choose a list of $2^r$ cube symmetries
\[
\tau_0,\tau_1,\dots,\tau_{2^r-1}\in W_I.
\]
These will be used \emph{in that order} to orient the copy of $C_{k-1}$ placed into
each subcube.

Write $N\coloneqq 2^{(k-1)r}$ (points per subcube).
Given a global time $t\in[0,2^{kr})$, split it into
\[
q\coloneqq \left\lfloor \frac{t}{N}\right\rfloor\in[0,2^r),
\qquad
s\coloneqq t\bmod N\in[0,N).
\]
Then $q$ selects \emph{which} subcube we are in, and $s$ is the local time within that
subcube.
Let $x_q\coloneqq I^{-1}(q)\in\V{r}$ be the corresponding corner.

The construction formula is
\begin{equation}\label{eq:construct}
C_k(t)\;=\; (\tau_q:I)\bigl(C_{k-1}(s)\bigr)\;+\; p^{(k)}(x_q).
\end{equation}

This is the same idea as the 2D Hilbert curve: visit the $2^r$ subcubes in some
corner-order, and inside each subcube run a rotated/reflected copy of the previous
curve.

\begin{definition}[Constructor relation (paper's Definition 1)]\label{def:constructor}
Let $C_{k-1}$ and $C_k$ be indexings of orders $k-1$ and $k$.
If \eqref{eq:construct} holds for some corner labeling $I$ and some
$\tau_0,\dots,\tau_{2^r-1}\in W_I$, we write
\[
C_{k-1}\ \xRightarrow[I]{(\tau_0,\dots,\tau_{2^r-1})}\ C_k,
\]
and we call $C_{k-1}$ a \emph{constructor} of $C_k$.
\end{definition}

\subsection{CSSIs and CHPs}

A \emph{CSSI} is what you get if you iterate the same construction rule at every level.

\begin{definition}[CSSI and CHP (paper's Definition 2)]\label{def:cssi}
A family $\mathcal C=\{C_k\mid k\ge 1\}$ is a \emph{class of self-similar indexings}
(\textbf{CSSI}) if there exist
\begin{itemize}[leftmargin=*,itemsep=2pt]
\item a base indexing $C_1$ of order $1$, and
\item a fixed list of symmetries $\tau_0,\dots,\tau_{2^r-1}\in W_{\widetilde C_1}$,
\end{itemize}
such that for every $k\ge 2$ we have
\[
C_{k-1}\ \xRightarrow[\widetilde C_1]{(\tau_0,\dots,\tau_{2^r-1})}\ C_k.
\]
Here $\widetilde C_1:\V{r}\to[0,2^r)$ is the \emph{canonical corner labeling} induced by
$C_1$ (it labels a corner by the time when $C_1$ visits it).

We write
\[
\mathcal H(C_1;\tau_0,\dots,\tau_{2^r-1})\coloneqq\{C_k\mid k\ge 1\}
\]
for the resulting CSSI, and we call $C_1$ the \emph{generator}.

A CSSI is a \emph{class with Hilbert property} (\textbf{CHP}) if every $C_k$ is a curve.
\end{definition}

\begin{remark}[``Check it in the small'']
Because the same local rule is used at every level, continuity questions are local.
In practice: once $C_2$ is continuous, the same seam pattern repeats at all larger scales,
so all $C_k$ will be continuous. This is the intuition behind the paper's emphasis that
large-scale properties can be detected from the small generating elements.
\end{remark}

\subsection{Example: the classical 2D Hilbert curve}

The standard 2D Hilbert curve is exactly a CHP obtained from:
\begin{itemize}[leftmargin=*,itemsep=2pt]
\item a generator $C_1$ that visits the four corners in a Gray-code order, and
\item four symmetries $\tau_0,\dots,\tau_3$ that rotate/reflect the copies in the four
quadrants so the pieces glue into one continuous path.
\end{itemize}

The paper records this as
$\mathcal H(\mathrm{Hil}^2_1;((2\,4),\id,\id,(1\,3)))=\{\mathrm{Hil}^2_k\mid k\ge 1\}$.
I will not reproduce their corner-label convention here; the important part is:
\emph{one generator + one fixed list of subcube symmetries completely specifies the whole family.}

\section{Disturbing a CSSI by a symmetry}

Section 3.2 of the paper is about a simple but powerful point:
if you ``disturb'' one level of a CSSI by a global cube symmetry, this disturbance
propagates through the recursion in a rigid algebraic way.

\subsection{How to read $(\varphi:I)$}

Fix a corner labeling $I$.
Given $\varphi\in W_I$, the paper writes $(\varphi:I)$ for the induced symmetry map
on the grid.
Composing a curve $C$ with $(\varphi:I)$ means: take every point visited by $C$ and
rotate/reflect the whole picture by that symmetry.

\subsection{Disturbing the constructor}

\begin{lemma}[Disturbing the constructor (paper's Lemma 1)]\label{lem:disturb-constructor}
Let $C_{k-1}$ and $C_k$ be indexings of orders $k-1$ and $k$.
Assume
\[
C_{k-1}\ \xRightarrow[I]{(\tau_0,\dots,\tau_{2^r-1})}\ C_k
\]
for some corner labeling $I$ and symmetries $\tau_q\in W_I$.

Then for every $\varphi\in W_I$,
\[
(\varphi:I)\circ C_{k-1}\ \xRightarrow[I]{(\tau_0\circ\varphi^{-1},\dots,\tau_{2^r-1}\circ\varphi^{-1})}\ C_k.
\]
\end{lemma}

\begin{proof}
Write the construction of $C_k$ as in \eqref{eq:construct}.
Inside subcube $q$ we have the map $(\tau_q:I)\circ C_{k-1}$.
If we replace $C_{k-1}$ by $(\varphi:I)\circ C_{k-1}$, then inside subcube $q$ we get
\[
(\tau_q:I)\circ (\varphi:I)\circ C_{k-1}
= ((\tau_q\circ\varphi):I)\circ C_{k-1}.
\]
To keep the same $C_k$, we therefore need to replace $\tau_q$ by
$\tau_q\circ\varphi^{-1}$, which gives exactly the claim.
(Algebraically this is just ``pulling'' $\varphi$ through the composition.)
\end{proof}

\subsection{Disturbing the corner labeling instead}

Lemma~\ref{lem:disturb-constructor} says: changing the constructor can be absorbed
by adjusting the $\tau$'s.
Lemma 2 says: you can instead keep the constructor but relabel corners and compare
with the globally transformed curve.

\begin{lemma}[Disturbing the corner labeling (paper's Lemma 2)]\label{lem:disturb-label}
Assume the setup of Lemma~\ref{lem:disturb-constructor}.
Fix $\varphi\in W_I$ and define a \emph{new} corner labeling
\[
K \coloneqq \varphi^{-1}\circ I.
\]
Let $\Phi$ denote the induced symmetry map on the grid (so $\Phi=(\varphi:I)$ in the
paper's notation; the point is that it is the same geometric symmetry no matter which
labeling you use).

Then
\[
C_{k-1}\ \xRightarrow[K]{(\tau_0\circ\varphi,\dots,\tau_{2^r-1}\circ\varphi)}\ \Phi\circ C_k.
\]
\end{lemma}

\begin{proof}[Proof idea]
Changing the labeling from $I$ to $K=\varphi^{-1}\circ I$ is just a relabeling of the
$2^r$ subcubes.
Under this relabeling, a permutation $\pi$ of labels acting under $I$ becomes the
conjugated permutation $\varphi^{-1}\circ \pi\circ \varphi$ under $K$.
If you apply this bookkeeping to the construction formula \eqref{eq:construct},
you get that transforming the \emph{whole} curve by $\Phi$ is equivalent to:
(i) changing the labeling to $K$, and (ii) multiplying each $\tau_q$ by $\varphi$.
\end{proof}

\subsection{A global consequence: symmetry shows up already in the generator}

\begin{theorem}[Symmetry at one level $\Leftrightarrow$ symmetry at all levels (paper's Theorem 3)]
\setcounter{theorem}{2} % make this Theorem 3
Let $\mathcal H(C_1;\tau_0,\dots,\tau_{2^r-1})=\{C_k\mid k\ge 1\}$ and
$\mathcal H(D_1;\tau_0,\dots,\tau_{2^r-1})=\{D_k\mid k\ge 1\}$ be two CSSIs
built with the same $\tau$-list (each interpreted under the canonical labeling of its
own generator).

Fix a cube symmetry $\varphi$ (formally $\varphi\in W_{\widetilde C_1}$) and let
$\Phi$ be the corresponding geometric symmetry of the grid.
Then the following are equivalent:
\begin{enumerate}[leftmargin=*,itemsep=2pt]
\item $\Phi\circ C_{k_0}=D_{k_0}$ for some $k_0\ge 1$,
\item $\Phi\circ C_k=D_k$ for all $k\ge 1$.
\end{enumerate}
\end{theorem}

\begin{proof}[Proof sketch]
(ii)$\Rightarrow$(i) is immediate.

For (i)$\Rightarrow$(ii): first observe that if the equality holds at some level $k_0$,
then it already forces the generators to match under the same symmetry,
$\Phi\circ C_1=D_1$, because the order-$k_0$ curve is built from $2^r$ copies of the
generator arranged along the generator's canonical corner labeling.

Then do induction on $k$.
Assume $D_k=\Phi\circ C_k$.
Use Lemma~\ref{lem:disturb-constructor} to rewrite the construction of $C_{k+1}$
in terms of a disturbed constructor, and use Lemma~\ref{lem:disturb-label} to
translate this disturbance to the construction of $\Phi\circ C_{k+1}$.
This shows $D_{k+1}=\Phi\circ C_{k+1}$.
\end{proof}

\begin{remark}[Why this matters]
Theorem~3 is the workhorse behind the classification results.
It says: if two self-similar families ever become the same up to symmetry, then they
were already the same (up to that symmetry) at the generator level.
So ``structural'' questions are reduced to a finite search over order-$1$ generators
and finite $\tau$-lists.
\end{remark}

\section{Applications to CHPs}

\subsection{Classification in 2D}

\begin{theorem}[Uniqueness of the 2D Hilbert curve (paper's Theorem 4)]
\setcounter{theorem}{3} % now Theorem 4
Up to global rotation/reflection (``modulo symmetry''), the classical 2D Hilbert curve
is the only 2D CHP.
\end{theorem}

\begin{proof}[What is being claimed / proof idea]
There are two finite checks:
\begin{itemize}[leftmargin=*,itemsep=2pt]
\item At order $1$ (a $2\times 2$ grid), the only continuous generator (up to symmetry)
is the obvious Gray-code path through the four corners.
\item Given that generator, you can brute-force (or reason combinatorially) about the
four subcube symmetries $\tau_0,\dots,\tau_3$ needed to build a continuous order-$2$
curve whose endpoints are corners. Only the classical Hilbert choice works.
\end{itemize}
Once order $2$ is fixed, self-similarity forces all higher orders.
\end{proof}

\subsection{``Simple'' CHPs and a basic restriction on generators}

For $r\ge 3$ the number of possible CHPs explodes, so the paper restricts attention to
a natural subclass.

\begin{definition}[Simple CHP]
A CHP is \emph{simple} if (for every $k$) the start- and endpoints of $C_k$ are corners
of $\Grid{k}{r}$.
Equivalently, the generator $C_1$ starts and ends at corners and the recursion preserves
this.
\end{definition}

\begin{remark}[Generators cannot start and end at opposite corners (paper's Remark 5)]
\setcounter{remark}{4} % make this Remark 5
Let $C_1$ be the generator of a simple $r$-dimensional CHP, and let $\widetilde C_1$
be its canonical corner labeling.
Then the first and last corners visited by $C_1$ are \emph{not} diagonally opposite:
\[
\hd\!\bigl(\widetilde C_1^{-1}(0),\,\widetilde C_1^{-1}(2^r-1)\bigr)\;<\; r.
\]
Intuition: if the generator's endpoints were opposite corners, then in an order-$2$
construction you would get stuck after placing only a couple of subcubes; there is no
way to glue $2^r$ copies continuously.
\end{remark}

\subsection{Classification in 3D}

\begin{theorem}[How many simple 3D CHPs exist (paper's Theorem 6)]
\setcounter{theorem}{5} % make this Theorem 6
In dimension $r=3$, there are exactly $6\cdot 2^8 = 1536$ structurally different
simple CHPs, where ``structurally different'' means \emph{not} identical up to a global
rotation/reflection.
\end{theorem}

\begin{proof}[Proof idea (as in the paper)]
Use Theorem~3 to reduce everything to order-$1$ data.
Then:
\begin{enumerate}[leftmargin=*,itemsep=2pt]
\item Enumerate continuous 3D generators modulo symmetry. The paper reports three
types (called $\mathrm{Hil}^{3,A}_1$, $\mathrm{Hil}^{3,B}_1$, $\mathrm{Hil}^{3,C}_1$).
\item For each generator type, enumerate all $\tau$-lists that produce a continuous
order-$2$ curve along the generator's canonical corner labeling.
\item Remark~5 rules out type $C$ as a generator for \emph{simple} CHPs.
\item For type $A$ there are $4$ distinct continuous ``arrangements''; for type $B$ there
are $2$.
In each of the $8$ subcubes of the order-$2$ cube there are $2$ orientation choices,
so each arrangement yields $2^8$ different CHPs.
\end{enumerate}
Total $=(4+2)\cdot 2^8 = 6\cdot 2^8$.
\end{proof}

\subsection{Constructing CHPs in higher dimensions}

The paper also sketches an inductive construction of at least one CHP in every
dimension:
roughly, you can build an $r$-dimensional \emph{generator} by joining two
$(r-1)$-dimensional generators and using the new dimension as a short connector.
(The paper gives a concrete 4D example with an explicit generator $\mathrm{Hil}^4_1$
and an explicit list of $16$ symmetries.)

I do not rewrite the 4D figure/table here, but the takeaway is:
\emph{their formalism makes it easy to talk about and compute with higher-dimensional
Hilbert-like curves once you have the generating elements.}

\subsection{Recursive computation}

Given a CSSI specified by $(C_1;\tau_0,\dots,\tau_{2^r-1})$, the recursion
\eqref{eq:construct} is already an algorithm.

\paragraph{Decoding: index $\to$ grid point.}
To compute $C_k(t)$ for a large $k$:
repeatedly split $t$ into $(q,s)$ (subcube number + local index),
apply the appropriate symmetry $\tau_q$ at that level, and add the offset $p^{(k)}(I^{-1}(q))$.

\paragraph{Encoding: grid point $\to$ index.}
The inverse map (point to index) is more delicate, but in practice it can be computed
by following the same recursion top-down: identify which subcube the point lies in,
undo that subcube's symmetry, recurse, and concatenate digits.
(Alber--Niedermeier mainly emphasize the structural recursion; later work focuses on
fast bit-level encoders.)

\section{Locality}

The last technical part of the paper reviews locality measures and gives explicit
constants for the 2D Hilbert curve.

\subsection{The $L_p$ locality measure}

Let $C:[0,n^r)\to\{0,\dots,n-1\}^r$ be a curve.
For $p\in\{1,2,\infty\}$ define
\[
L_p(C)\coloneqq \max_{\substack{i,j\in[0,n^r)\\ i\neq j}}
\frac{ d_p\!\bigl(C(i),C(j)\bigr)^r}{|i-j|}.
\]
This is the paper's definition (their Equation (2) is the $p=2$ case).

\subsection{Explicit bounds for 2D Hilbert}

\begin{theorem}[Uniform locality bounds in 2D (paper's Theorem 7)]
\setcounter{theorem}{6} % make this Theorem 7
Let $\mathcal H^2=\{\mathrm{Hil}^2_k\mid k\ge 1\}$ be the classical 2D Hilbert CHP.
Then for every order $k\ge 1$,
\[
L_1(\mathrm{Hil}^2_k)\le 9\frac{3}{5},\qquad
L_2(\mathrm{Hil}^2_k)\le 6\frac{1}{2},\qquad
L_\infty(\mathrm{Hil}^2_k)\le 6\frac{2}{5}.
\]
\end{theorem}

\begin{proof}[Proof idea (high level)]
Fix two indices $i<j$ and set $s=j-i$.
Choose a scale $m$ so that $4^{m-1}< s\le 4^m$ (so $s$ spans about $m$ blocks at the
next coarser Hilbert scale).
Now subdivide the $2^k\times 2^k$ grid into subgrids of side length $2^{m-1}$ and use
self-similarity to argue that the subpath from $i$ to $j$ must live in a sequence of
$\ell$ or $\ell+1$ such subgrids (for a small $\ell$ depending on $m$).
The crucial point is that in the 2D Hilbert curve, there are only finitely many possible
patterns for such ``subgrid sequences'' up to symmetry.
You can therefore enumerate worst cases and obtain uniform upper bounds for the
ratios $d_p^2/|i-j|$.
The paper carries this out (with a further refinement step at scale $2^{m-2}$) and
extracts the constants above.
\end{proof}

\section{What I personally take away}

\begin{itemize}[leftmargin=*,itemsep=2pt]
\item A ``Hilbert-like'' multidimensional curve is \emph{not} a single object; in 3D there
are many non-equivalent ones. The generator+symmetry formalism makes this explicit.
\item Theorem~3 is the conceptual core: in a rigid self-similar family, global symmetry
behavior is already visible at the generator level.
\item This directly enables classification-by-search in low dimensions (2D unique, 3D
has $1536$ simple variants).
\item For analysis tasks like locality, self-similarity lets you reduce to finitely many
small patterns (the proof strategy behind Theorem~7).
\end{itemize}

\end{document}
